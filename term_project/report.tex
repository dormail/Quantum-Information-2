% ****** Start of file apssamp.tex ******
%
%   This file is part of the APS files in the REVTeX 4.2 distribution.
%   Version 4.2a of REVTeX, December 2014
%
%   Copyright (c) 2014 The American Physical Society.
%
%   See the REVTeX 4 README file for restrictions and more information.
%
% TeX'ing this file requires that you have AMS-LaTeX 2.0 installed
% as well as the rest of the prerequisites for REVTeX 4.2
%
% See the REVTeX 4 README file
% It also requires running BibTeX. The commands are as follows:
%
%  1)  latex apssamp.tex
%  2)  bibtex apssamp
%  3)  latex apssamp.tex
%  4)  latex apssamp.tex
%
\documentclass[%
reprint,
%superscriptaddress,
%groupedaddress,
%unsortedaddress,
%runinaddress,
%frontmatterverbose, 
%preprint,
%preprintnumbers,
%nofootinbib,
%nobibnotes,
%bibnotes,
amsmath,amssymb,
aps,
%pra,
%prb,
%rmp,
%prstab,
%prstper,
%floatfix,
]{revtex4-2}

\usepackage{graphicx}% Include figure files
\usepackage{float}
\usepackage{dcolumn}% Align table columns on decimal point
\usepackage{bm}% bold math
\usepackage{hyperref}% add hypertext capabilities
\usepackage{caption}
\usepackage{subcaption}
\usepackage{tikz}
\usepackage{braket}
%\usepackage[mathlines]{lineno}% Enable numbering of text and display math
%\linenumbers\relax % Commence numbering lines

%\usepackage[showframe,%Uncomment any one of the following lines to test 
%%scale=0.7, marginratio={1:1, 2:3}, ignoreall,% default settings
%%text={7in,10in},centering,
%%margin=1.5in,
%%total={6.5in,8.75in}, top=1.2in, left=0.9in, includefoot,
%%height=10in,a5paper,hmargin={3cm,0.8in},
%]{geometry}

%% my own packages
\usepackage{siunitx}

\begin{document}

\title{Project report: Non-adiabatic holonomic quantum computation in NV-Diamond}% Force line breaks with \\
%\thanks{I%A footnote to the article title}%

\author{Matthias Maile}
\email{matthias.maile@kaist.ac.kr}
\affiliation{
  Korea Advanced Institute of Science and Technology
}

\date{\today}% It is always \today, today,
%  but any date may be explicitly specified

%\begin{abstract}
%  In this final report the results of the project for the course
% Computational Physics at KAIST get summarized, along a description of the used methods
%  and technologies used. In the project, a low level implementation in C++ has been done, achieving
%  fast runtime and high flexibillity. With the completed implementation the symmetric (`American`) 
%  traffic law gets compared to the widely used asymmetric driving law regarding passing on left
%  lanes. The simulation is based on state-of-the-art traffic models with the \textit{intelligent driver
%  model} (IDM) for in-lane dynamics and the \textit{minimizing overal braking induced by lane changes}
%  (MOBIL) model as a decision tree for lane changes.
%\end{abstract}

%\keywords{Suggested keywords}%Use showkeys class option if keyword
%display desired
\maketitle

\tableofcontents

\section{Introduction}
\label{sec:Introduction}
In this project we reproduce the theoretical part of a paper covering an experimental
implementation of non-adiabatic holonomic quantum computing \cite{PhysRevApplied.16.024060}. 
As a platform the group used a diamond with a nitrogen-vacancy (NV) defect. This report will give 
introduction about the advantages of using this NV-diamond as a qubit(\autoref{sec:intr:nv-diamond})
and holonomic quantum computing \autoref{sec:HQC}. We will explain the original methodology and our
own implementation \autoref{sec:Methodology} and our results \autoref{sec:Results}.

\subsection{Nitrogen-Vacancy Diamond}
\label{sec:intr:nv-diamond}
The experimental work has been done in a solid state diamond with a nitrogen-vacancy center, shownm
in \autoref{fig:NV-center} on the left. A useful feature is the photoluminescence, through which the
spin state can be measured. Energywise, the system can be treated as a triplet with three states
$\ket{0}$, $\ket{1}$ and $\ket{a}$. The structure is shown in \autoref{fig:NV-center} on the right.

\begin{figure}
  \centering
  \includegraphics[width=0.45\textwidth]{media/cell_energy.png}
  \caption{Left: Schematic of the unit cell of diamond, includ-
    ing an N-V center. Right: Encoding of a qubit in the spin-triplet
  ground state and microwave coupling configuration. Both taken from original paper
\cite{PhysRevApplied.16.024060}.}
  \label{fig:NV-center}
\end{figure}

  \subsection{Holonomic Quantum Computating}
  \label{sec:HQC}

  \section{Methodology}
  \label{sec:Methodology}

  \section{Results}
  \label{sec:Results}

  \begin{figure}
    \centering
    \begin{subfigure}[b]{0.45\textwidth}
      \centering
      \includegraphics[width=1\textwidth]{build/1-Omega.pdf}
      \caption{$\Omega(t)$}
      \label{fig:1-Omega}
    \end{subfigure}
    \,
    \begin{subfigure}[b]{0.45\textwidth}
      \centering
      \includegraphics[width=1\textwidth]{build/1-phi_1.pdf}
      \caption{$\phi_1(t)$}
      \label{fig:1-phi_1}
    \end{subfigure}
    \caption{Amplitude and phase of the laser.}
  \end{figure}



  \begin{figure}
    \centering
    \begin{subfigure}[b]{0.45\textwidth}
      \centering
      \includegraphics[width=1\textwidth]{build/2-Omega.pdf}
      \caption{$\Omega(t)$}
      \label{fig:1-Omega}
    \end{subfigure}
    \,
    \begin{subfigure}[b]{0.45\textwidth}
      \centering
      \includegraphics[width=1\textwidth]{build/2-phi_1.pdf}
      \caption{$\phi_1(t)$}
      \label{fig:1-phi_1}
    \end{subfigure}
    \caption{Amplitude and phase of the laser, according to theory.}
  \end{figure}

  \begin{figure}
    \centering
    \includegraphics[width=0.45\textwidth]{media/Fig5ab.png}
    \caption{Amplitude and phase of MW field used in experiments for the NHQC+ X gate.}
  \end{figure}
  \begin{figure}
    \centering
    \includegraphics[width=0.45\textwidth]{media/Fig6.png}
    \caption{Amplitude and phase of MW field used in experiments for the NHQC+ $X$ and $X/2$ gate,
    $\eta = 0.4$.}
  \end{figure}

  \begin{figure}
    \centering
    \includegraphics[width=0.45\textwidth]{build/1-probabilities.pdf}
    \caption{Probability of bright, dark and $a$ state under the $X/2$ gate with $\eta=1$.}
  \end{figure}

  \begin{figure}
    \centering
    \includegraphics[width=0.45\textwidth]{media/probabilities.png}
    \caption{Holonomic X gate on a bright state. Experimental results
    (dots) fit well with simulations (solid lines) with η = 1. Taken from the original author`s paper.}
  \end{figure}

  \begin{figure}
    \centering
    \includegraphics[width=0.45\textwidth]{build/2-probabilities.pdf}
    \caption{Probabilites of $\ket{0}$, $\ket{1}$ and $\ket{a}$ states when evolving a $\ket{0}$
    state under the $X$ gate.}
  \end{figure}

  \begin{figure}
    \centering
    \includegraphics[width=0.45\textwidth]{build/error.pdf}
    \caption{Fidelity of the $X$ gate when replacing the MW amplitude $\Omega \rightarrow
    (1+\alpha)\Omega$.}
  \end{figure}


  \begin{figure}
    \centering
    \includegraphics[width=0.45\textwidth]{media/error.png}
    \caption{Performance of an X gate with control errors (α) for NHQC and
      NHQC+ schemes with initial state |0␂. Envelopes of the two
    driving fields of NHQC are truncated Gaussian pulses. }
  \end{figure}

  \section{Conclusion}
  \label{sec:Introduction}

  \appendix


  % The \nocite command causes all entries in a bibliography to be printed out
  % whether or not they are actually referenced in the text. This is appropriate
  % for the sample file to show the different styles of references, but authors
  % most likely will not want to use it.
  \nocite{*}

  %\bibliography{apssamp}% Produces the bibliography via BibTeX.
  \bibliography{lit}% Produces the bibliography via BibTeX.

  \end{document}
  %
  % ****** End of file apssamp.tex ******
