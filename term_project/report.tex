% ****** Start of file apssamp.tex ******
%
%   This file is part of the APS files in the REVTeX 4.2 distribution.
%   Version 4.2a of REVTeX, December 2014
%
%   Copyright (c) 2014 The American Physical Society.
%
%   See the REVTeX 4 README file for restrictions and more information.
%
% TeX'ing this file requires that you have AMS-LaTeX 2.0 installed
% as well as the rest of the prerequisites for REVTeX 4.2
%
% See the REVTeX 4 README file
% It also requires running BibTeX. The commands are as follows:
%
%  1)  latex apssamp.tex
%  2)  bibtex apssamp
%  3)  latex apssamp.tex
%  4)  latex apssamp.tex
%
\documentclass[%
reprint,
%superscriptaddress,
%groupedaddress,
%unsortedaddress,
%runinaddress,
%frontmatterverbose, 
%preprint,
%preprintnumbers,
%nofootinbib,
%nobibnotes,
%bibnotes,
amsmath,amssymb,
aps,
%pra,
%prb,
%rmp,
%prstab,
%prstper,
%floatfix,
]{revtex4-2}

\usepackage{graphicx}% Include figure files
\usepackage{float}
\usepackage{dcolumn}% Align table columns on decimal point
\usepackage{bm}% bold math
\usepackage{hyperref}% add hypertext capabilities
\usepackage{caption}
\usepackage{subcaption}
\usepackage{tikz}
\usepackage{braket}
%\usepackage[mathlines]{lineno}% Enable numbering of text and display math
%\linenumbers\relax % Commence numbering lines

%\usepackage[showframe,%Uncomment any one of the following lines to test 
%%scale=0.7, marginratio={1:1, 2:3}, ignoreall,% default settings
%%text={7in,10in},centering,
%%margin=1.5in,
%%total={6.5in,8.75in}, top=1.2in, left=0.9in, includefoot,
%%height=10in,a5paper,hmargin={3cm,0.8in},
%]{geometry}

%% my own packages
\usepackage{siunitx}

\begin{document}

\title{Project report: Non-adiabatic holonomic quantum computation in NV-Diamond}% Force line breaks with \\
%\thanks{I%A footnote to the article title}%

\author{Matthias Maile}
\email{matthias.maile@kaist.ac.kr}
\affiliation{
  Korea Advanced Institute of Science and Technology
}

\date{\today}% It is always \today, today,
%  but any date may be explicitly specified

\begin{abstract}
  In this project we simulated a three-level system in the context of non-adiabatic holonomic quantum
  computing. We reproduced a few figures from an experimental paper in our own simulation. In this
  document we provide our calculations and figures and the code is provided in a python file.
\end{abstract}

%\keywords{Suggested keywords}%Use showkeys class option if keyword
%display desired
\maketitle

\tableofcontents

\section{Introduction}
\label{sec:Introduction}
In this project we reproduce the theoretical part of a paper covering an experimental
implementation of non-adiabatic holonomic quantum computing \cite{PhysRevApplied.16.024060}. 
As a platform the group used a diamond with a nitrogen-vacancy (NV) defect. This report will give 
introduction about the advantages of using this NV-diamond as a qubit(\autoref{sec:intr:nv-diamond})
and holonomic quantum computing \autoref{sec:HQC}. We will explain the original methodology and our
own implementation \autoref{sec:Methodology} and our results \autoref{sec:Results}.

\subsection{Nitrogen-Vacancy Diamond}
\label{sec:intr:nv-diamond}
The experimental work has been done in a solid state diamond with a nitrogen-vacancy center, shownm
in \autoref{fig:NV-center} on the left. A useful feature is the photoluminescence, through which the
spin state can be measured. Energywise, the system can be treated as a triplet with three states
$\ket{0}$, $\ket{1}$ and $\ket{a}$. The structure is shown in \autoref{fig:NV-center} on the right.

\begin{figure}
  \centering
  \includegraphics[width=0.45\textwidth]{media/cell_energy.png}
  \caption{Left: Schematic of the unit cell of diamond, includ-
    ing an N-V center. Right: Encoding of a qubit in the spin-triplet
    ground state and microwave coupling configuration. Both taken from original paper
  \cite{PhysRevApplied.16.024060}.}
  \label{fig:NV-center}
\end{figure}

\subsection{Holonomic Quantum Computating}
\label{sec:HQC}
Holonomic quantum computing has been proposed theoretically in 1999 \cite{ZANARDI199994}.

What set holonomic quantum computing apart was the usage of geometric phases for quantum information
processing. These geometric phases, or generalized Berry phases, are phases that arise due to
different different trajectories in the hilbert space, in the case of two-level qubits this can be
imagined as trajectories on the bloch sphere.

A simple example of this phase is a s qubit starting in an eigenstate of the Pauli-$z$ matrix. A
rotation around the x-axis, followed by a rotation around the y-axis brings it back to the original
state, but a phase factor arises. The original authors showed it in a cute figure shown in
\autoref{fig:holonomy}.

\begin{figure}
  \centering
  \includegraphics[width=0.25\textwidth]{media/holonomic phase.png}
  \caption{Conceptual explanation of geometric quantum operation in the Bloch
sphere, showing the geometric phase that equals half of the
enclosed solid angle, $\gamma_G$. Taken from the original paper.} 
  \label{fig:holonomy}
\end{figure}

A challenge with holonomic quantum computing was the dependency on adiabatic processes, which by
design need long run times. This can prevent a quantum advantage since these long runtimes lead to
decoherence and loss of quantumness in the system. The authors overcame this in the grand scheme of
non-adiabatic quantum computing (NHQC) and tried to improve it further in the paper we will refer to
a lot in the following sections. This new, supposedly improved method is called NHQC+ which we will
also use throughout this report.

\section{Methodology}
\label{sec:Methodology}
The system can be described by a Rabi-inspired hamiltonian 
\begin{align*}
  H_1 &= \left[
    \frac{\Omega_1(t)}{2} e^{-i\phi_1(t)} \ket{0} + \frac{\Omega_2(t)}{2} e^{-i\phi_2(t)}
  \ket{1} \right] \bra{a} + H.c. \\
      &= \frac{\Omega}{2} e^{-i\phi_1} \ket{b}\bra{a} + H.c.
\end{align*}
where
\begin{itemize}
  \item bright state is defined as $\ket{b} = \sin\frac{\theta}{2} \ket{0} - \cos\frac{\theta}{2}
    e^{i\phi} \ket{1}$,
  \item $\theta$ and $\phi$ are time independent parameters controlling the gate operation,
  \item $\Omega = \sqrt{\Omega_1^2 + \Omega_2^2}$ is the Rabi frequency.
\end{itemize}
You can defined two more states
\[
  \ket{d} = \cos\frac{\theta}{2} \ket{0} + \sin\frac{\theta}{2} e^{i\phi} \ket{1}
\]
\[
  \ket{\psi} = e^{if/2} \left[ \cos\frac{\beta}{2} e^{-i\varphi/2} \ket{b} + \sin\frac{\beta}{2}
  e^{i\varphi/2} \ket{a} \right]
\]
with the time dependent parameters $f$, $\beta$ and $\varphi$. Using the von-Neumann equation for
each projector of these states, the following system of equation arises:
\begin{align*}
  \dot f        &= \frac{\dot \varphi}{\cos \beta} \\
  \dot \beta    &= \Omega \sin(\varphi + \phi_1)  \\
  \dot \varphi  &= \Omega \cot\beta \cos(\varphi + \phi_1).
\end{align*}
A solution turns out to be
\begin{align*}
  \beta   &= \pi \sin^2\left(\frac{\pi t}{\tau} \right) \\
  f       &= \eta \left[2\beta - \sin(2\beta) \right] \\
  \varphi &= \eta \left[\sin(\beta) - \frac{1}{3} \sin(3\beta) \right],
\end{align*}
however it is not unique as other solutions are possible
\cite{https://doi.org/10.1002/qute.202000001}.

Then the phase and amplitude of the external field in the hamiltonian is given by
\begin{align}
  \phi_1 &= \arctan\left(\frac{\dot \beta}{\dot \varphi} \cot\beta\right) - \varphi +
  \text{const.} \\
  \Omega &= \frac{\dot \beta}{\sin(\varphi + \phi_1)}
  \label{eqn:params}
\end{align}

This calculation gives as a timedependent hamiltonian in the form of a $3\times 3$ matrix, which can
be inserted into the Schroedinger equation. The later can be integrated with numerical method, where
we chose Euler`s method with $\Delta t = \SI{0.1}{ns}$ for simplicity. 

The simulation is implemented in python with the numpy library. The code is attached and run inside
a jupyter notebook.

\section{Results}
\label{sec:Results}

\subsection{Reproducing $\Omega(t)$ and $\phi_1(t)$}
\label{sec:ReproducingParameters}
The main challenge was reproducing the parameters in the hamiltonian given by \autoref{eqn:params}.
After the theoretical derivation and correct boundry conditions I was able to reproduce the Figures
5 and 6 from the original authors. A sample of these is shown in \autoref{fig:params-own1} and
\autoref{fig:params-own2}.
The original work is presented in \autoref{sec:Amplitude and phase of EM-field given in the original paper}.
\begin{figure}
  \centering
  \begin{subfigure}[b]{0.45\textwidth}
    \centering
    \includegraphics[width=1\textwidth]{build/1-Omega.pdf}
    \caption{$\Omega(t)$}
    \label{fig:1-Omega}
  \end{subfigure}
  \,
  \begin{subfigure}[b]{0.45\textwidth}
    \centering
    \includegraphics[width=1\textwidth]{build/1-phi_1.pdf}
    \caption{$\phi_1(t)$}
    \label{fig:1-phi_1}
  \end{subfigure}
  \caption{Amplitude and phase of the laser to produce an $X/2$ gate. Contrarry to the given plots in
  \cite{PhysRevApplied.16.024060} we set $\eta=1$.}
  \label{fig:params-own1}
\end{figure}


\begin{figure}
  \centering
  \begin{subfigure}[b]{0.45\textwidth}
    \centering
    \includegraphics[width=1\textwidth]{build/2-Omega.pdf}
    \caption{$\Omega(t)$}
    \label{fig:1-Omega}
  \end{subfigure}
  \,
  \begin{subfigure}[b]{0.45\textwidth}
    \centering
    \includegraphics[width=1\textwidth]{build/2-phi_1.pdf}
    \caption{$\phi_1(t)$}
    \label{fig:1-phi_1}
  \end{subfigure}
  \caption{Amplitude and phase of the laser to create an $X$ gate with $\eta=1$.}
  \label{fig:params-own2}
\end{figure}

\subsection{Applying a holonomic X/2 gate on a bright state}
\label{sec:Applying a holonomic gate on a bright state}
The original authors decided to apply a $X/2$ gate on the bright state $\ket{b}$, which in this
context is a $\ket{+}$ state. Here the geometric phase becomes visible through a phase 
appearing when looking at the simulated final state. Since this phase is purely geometric, the
statistics of the computational basis states is the same as with the initial state. Our result for
the probabilities is shown in \autoref{fig:X/2}. This is inspired by a figure in the paper shown in 
\autoref{fig:X:orig}.
\begin{figure}
  \centering
  \includegraphics[width=0.45\textwidth]{build/1-probabilities.pdf}
  \caption{Probability of bright, dark and $a$ state under the $X/2$ gate with $\eta=1$.}
  \label{fig:X/2}
\end{figure}

\subsection{X (NOT) Gate}
\label{sec:X (NOT) Gate}
For universal quantum computing a NOT gate is useful, so it has also been applied in the $\{
\ket{0}, \ket{1} \}$ basis. This gives a good tool to measure the accuracy and error resiliance of
the method. The probability evolution is shown in \autoref{fig:X-probs}. 
\begin{figure}
  \centering
  \includegraphics[width=0.45\textwidth]{build/2-probabilities.pdf}
  \caption{Probabilites of $\ket{0}$, $\ket{1}$ and $\ket{a}$ states when evolving a $\ket{0}$
  state under the $X$ gate.}
  \label{fig:X-probs}
\end{figure}
The error resilience can be tested by replacing the amplitude in the hamiltonian 
\[
  \Omega \rightarrow (1 + \alpha) \Omega
\]
where $\alpha$ corresponds to the error.

The fidelity can be defined as the probabilty to find the system in $\ket{1}$ after the operation
\[
  F = 
  \vert
  \bra{\psi_\text{fin}}
  \ket{0}
  \vert^2
\]
the function $F(\alpha)$ is shown in \autoref{fig:fid_theory}.
\begin{figure}[H]
  \centering
  \includegraphics[width=0.45\textwidth]{build/error.pdf}
  \caption{Fidelity of the $X$ gate when replacing the MW amplitude $\Omega \rightarrow
  (1+\alpha)\Omega$.}
  \label{fig:fid_theory}
\end{figure}



\section{Conclusion}
\label{sec:Introduction}
The original aim of this project, to reproduce \autoref{fig:X:orig} has been achieved, as it is
apparent through \autoref{fig:X/2}. Afterwards, we also reproduced the error resiliant behavior of
the NHQC+ scheme. 

Further time can be spent on multi qubit gates, which were left out of scope since the paper did not
contain enough information on how this method is generalized from a single qubit case.

\appendix

\section{Amplitude and phase of EM-field given in the original paper}
\label{sec:Amplitude and phase of EM-field given in the original paper}
The authors did not provide the full equations for $\Omega$ and $\phi_1$ and instead opted to give
some figures, which I reproduced in \autoref{sec:Results}. Their figures are shown in
\autoref{fig:original1} and
\autoref{fig:original2}. The visual difference come from the authors using $\eta = 0.4$ in
\autoref{fig:original2}, eventhough for the whole experimental implementation they set $\eta=1$ since it
is a direct tool to supress errors. 

\begin{figure}
  \centering
  \includegraphics[width=0.45\textwidth]{media/Fig5ab.png}
  \caption{Amplitude and phase of MW field used in experiments for the NHQC+ X gate.}
  \label{fig:original1}
\end{figure}
\begin{figure}
  \centering
  \includegraphics[width=0.45\textwidth]{media/Fig6.png}
  \caption{Amplitude and phase of MW field used in experiments for the NHQC+ $X/2$ and $X$ gate,
  $\eta = 0.4$.}
  \label{fig:original2}
\end{figure}

\section{$X/2$ gate on a bright state}
\label{sec:$X/2$ gate}
The probabilities during the $X/2$ operation with the bright state (here $\ket{b} = \ket{+}$) and
theoretical curves are
shown in \autoref{fig:X:orig}.
\begin{figure}
  \centering
  \includegraphics[width=0.45\textwidth]{media/probabilities.png}
  \caption{Holonomic X gate on a bright state. Experimental results
  (dots) fit well with simulations (solid lines) with $\eta = 1$. Taken from the original author`s paper.}
  \label{fig:X:orig}
\end{figure}

\section{Fidelity in the experiment}
\label{sec:FidelityExp}
The authors also tested the fidelity in the experiment, the results are shown in 
\autoref{fig:fidelity_exp} where we only care about the upper/orange line since the authors compare
their new method to an older implementation of non-adiabatic quantum computation.

\begin{figure}
  \centering
  \includegraphics[width=0.45\textwidth]{media/error.png}
  \caption{Performance of an X gate with control errors ($\alpha$) for NHQC and
    NHQC+ schemes with initial state $\ket{0}$. Envelopes of the two
  driving fields of NHQC are truncated Gaussian pulses. }
  \label{fig:fidelity_exp}
\end{figure}

% The \nocite command causes all entries in a bibliography to be printed out
% whether or not they are actually referenced in the text. This is appropriate
% for the sample file to show the different styles of references, but authors
% most likely will not want to use it.
\nocite{*}

%\bibliography{apssamp}% Produces the bibliography via BibTeX.
\bibliography{lit}% Produces the bibliography via BibTeX.

\end{document}
%
% ****** End of file apssamp.tex ******
