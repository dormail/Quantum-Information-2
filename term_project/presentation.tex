% This example is meant to be compiled with lualatex or xelatex
% The theme itself also supports pdflatex
\PassOptionsToPackage{unicode}{hyperref}
\documentclass[aspectratio=1610, 9pt]{beamer}

% Load packages you need here
\usepackage{polyglossia}
\setmainlanguage{english}

\usepackage{csquotes}


\usepackage{amsmath}
\usepackage{amssymb}
\usepackage{mathtools}

\usepackage{hyperref}
\usepackage{bookmark}

% load the theme after all packages
\usepackage{tikz}
\usepackage{braket}

\usetheme[
showtotalframes, % show total number of frames in the footline
% dark, % optional dark theme, uncomment to use
]{tudo}

% Put settings here, like
\unimathsetup{
  math-style=ISO,
  bold-style=ISO,
  nabla=upright,
  partial=upright,
  mathrm=sym,
}

\title{Quantum Computation on
Solid-State Spins with Optimal Control - Term Project on Solid State Qubits}
\author[M.~Maile]{Matthias Maile}
\institute[TU Dortmund, KAIST]{TU Dortmund, KAIST}
\titlegraphic{\includegraphics[width=0.23\textwidth]{media/Nitrogen-vacancy_center.png}}

% extra packages

% for subfigs
\usepackage{caption}
\usepackage{subcaption}

\begin{document}

\maketitle

\section{Introduction}
\label{sec:Introduction}

\begin{frame}{NV-Diamond}
\begin{block}{NV-Diamond}
  Physical system: Diamond with Nitrogen-Vacancy modification
  \begin{itemize}
    \item well studied imperfection of diamond
    \item Photoluminescence: spin state of NV-center can be measured
  \end{itemize}
\end{block}
\end{frame}

\begin{frame}
   \begin{figure}
    \centering
    \includegraphics[height=0.5\textheight]{media/cell_energy.png}
    \caption{Left: Schematic of the unit cell of diamond, includ-
ing an N-V center. Right: Encoding of a qubit in the spin-triplet
ground state and microwave coupling configuration. Both taken from original paper.}
    \label{fig:NV-center}
  \end{figure}
\end{frame}

\section{Theoretical description}
\label{sec:theory}

\begin{frame}{}
  \begin{block}{Hamiltonian}
    \begin{align*}
      H_1 &= \left[
        \frac{\Omega_1(t)}{2} e^{-i\phi_1(t)} \ket{0} + \frac{\Omega_2(t)}{2} e^{-i\phi_2(t)}
      \ket{1} \right] \bra{a} + H.c. \\
          &= \frac{\Omega}{2} e^{-i\phi_1} \ket{b}\bra{a} + H.c.
    \end{align*}
    Where
    \begin{itemize}
      \item bright state $\ket{b} = \sin\frac{\theta}{2} \ket{0} - \cos\frac{\theta}{2}
      e^{i\phi} \ket{1}$, 
    \item $\theta$ and $\phi$ time independent parameters controlling gate operation,
    \item $\Omega = \sqrt{\Omega_1^2 + \Omega_2^2}$ Rabi frequency
    \end{itemize}
  \end{block}
\end{frame}

\begin{frame}{Finding $\Omega(t)$ and $\phi_1(t)$}
    $\Omega(t)$ and $\phi_1(t)$ are amplitude and phase of our laser, only time dependent parameters
    in Hamiltonian.
    \begin{block}{more parameters}
      The authors introduce two more states
      \[
        \ket{d} = \cos\frac{\theta}{2} \ket{0} + \sin\frac{\theta}{2} e^{i\phi} \ket{1}
        \qquad
        \ket{\psi} = e^{if/2} \left[ \cos\frac{\beta}{2} e^{-i\varphi/2} \ket{b} + \sin\frac{\beta}{2} 
          e^{i\varphi/2} \ket{a} \right]
      \]
      which depend on $f$, $\beta$ and $\varphi$, 
      their respective projectors conform to the \alert{von
      Neumann equation}.

      This gives the system of equations
      \begin{align*}
        \dot f        &= \frac{\dot \varphi}{\cos \beta} \\
        \dot \beta    &= \Omega \sin(\varphi + \phi_1)  \\
        \dot \varphi  &= \Omega \cot\beta \cos(\varphi + \phi_1)
      \end{align*}
    \end{block}
\end{frame}

\begin{frame}
  The system can be solved
  \begin{align*}
    \beta   &= \pi \sin^2\left(\frac{\pi t}{\tau} \right) \\
    f       &= \eta \left[2\beta - \sin(2\beta) \right] \\
    \varphi &= \eta \left[\sin(\beta) - \frac{1}{3} \sin(3\beta) \right]
  \end{align*}

  Then 
  \begin{align*}
    \phi_1 &= \arctan\left(\frac{\dot \beta}{\dot \varphi} \cot\beta\right) - \varphi +
    \text{const.} \\
    \Omega &= \frac{\dot \beta}{\sin(\varphi + \phi_1)}
  \end{align*}
\end{frame}

\section{Theoretical Results}
\label{sec:Theoretical Results}

\begin{frame}{$\Omega$, $\phi_1$ to create a $X/2$ gate}
  \begin{figure}
    \centering
    \begin{subfigure}[b]{0.45\textwidth}
      \centering
      \includegraphics[width=1\textwidth]{build/1-Omega.pdf}
      \caption{$\Omega(t)$}
      \label{fig:1-Omega}
    \end{subfigure}
    \,
    \begin{subfigure}[b]{0.45\textwidth}
      \centering
      \includegraphics[width=1\textwidth]{build/1-phi_1.pdf}
      \caption{$\phi_1(t)$}
      \label{fig:1-phi_1}
    \end{subfigure}
    \caption{Amplitude and phase of the laser.}
  \end{figure}
\end{frame}

\begin{frame}{$\Omega$, $\phi_1$ to create a $X$ gate}
  \begin{figure}
    \centering
    \begin{subfigure}[b]{0.45\textwidth}
      \centering
      \includegraphics[width=1\textwidth]{build/2-Omega.pdf}
      \caption{$\Omega(t)$}
      \label{fig:1-Omega}
    \end{subfigure}
    \,
    \begin{subfigure}[b]{0.45\textwidth}
      \centering
      \includegraphics[width=1\textwidth]{build/2-phi_1.pdf}
      \caption{$\phi_1(t)$}
      \label{fig:1-phi_1}
    \end{subfigure}
    \caption{Amplitude and phase of the laser, according to theory.}
  \end{figure}
\end{frame}

\begin{frame}{Original author`s work}
  \begin{figure}
    \centering
      \includegraphics[height=.35\textheight]{media/Fig5ab.png}
      \caption{Amplitude and phase of MW field used in experiments for the NHQC+ X gate.}
  \end{figure}
  \begin{figure}
    \centering
      \includegraphics[height=.35\textheight]{media/Fig6.png}
      \caption{Amplitude and phase of MW field used in experiments for the NHQC+ $X$ and $X/2$ gate,
      $\eta = 0.4$.}
  \end{figure}
\end{frame}

\begin{frame}{Applying the gates on quantum states}
  We can apply the $X/2$ gate on $\ket{b} = \ket{0} - \ket{1}$. The final state has a
  geometric phase $\ket{\psi_\text{fin}} = i \ket{b}$.
  \begin{figure}
    \centering
      \includegraphics[height=.8\textheight]{build/1-probabilities.pdf}
      \caption{Probability of bright, dark and $a$ state under the $X/2$ gate with $\eta=1$.}
  \end{figure}
\end{frame}

\begin{frame}{Similar work from the author}
  \begin{figure}
    \centering
      \includegraphics[height=.8\textheight]{media/probabilities.png}
      \caption{Holonomic X gate on a bright state. Experimental results
(dots) fit well with simulations (solid lines) with η = 1. Taken from the original author`s paper.}
  \end{figure}
\end{frame}

\begin{frame}{$X$ Gate}
  \begin{figure}
    \centering
      \includegraphics[height=.8\textheight]{build/2-probabilities.pdf}
      \caption{Probabilites of $\ket{0}$, $\ket{1}$ and $\ket{a}$ states when evolving a $\ket{0}$
      state under the $X$ gate.}
  \end{figure}
\end{frame}

\begin{frame}{Error Resilience}
  We can test how the fidelity  evolves when the amplitude of the MW field is not exact:
  \begin{figure}
    \centering
      \includegraphics[height=.8\textheight]{build/error.pdf}
      \caption{Fidelity of the $X$ gate when replacing the MW amplitude $\Omega \rightarrow
      (1+\alpha)\Omega$.}
  \end{figure}
\end{frame}

\begin{frame}{Experimental Test}
  \begin{figure}
    \centering
      \includegraphics[height=.8\textheight]{media/error.png}
      \caption{Performance of an X gate with control errors (α) for NHQC and
NHQC+ schemes with initial state |0␂. Envelopes of the two
driving fields of NHQC are truncated Gaussian pulses. }
  \end{figure}
\end{frame}

\section{Conclusion}
\label{sec:Conclusion}

\begin{frame}{Observations}
  \begin{itemize}
    \item The derivation of optimal parameters $\Omega$ and $\phi_1$ is feasible
    \item Under good settings the error is low and predictable
    \item Noise can be reduced such that the system behaves according to the theoretical
      description
  \end{itemize}
\end{frame}

\end{document}
